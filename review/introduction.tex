% !Mode:: "TeX:UTF-8"
\chapter{概述}

\section{问题背景}
直升机以其特有的垂直起降(Vertical Take-Off and Landing, VTOL)、悬停和低空低速飞行能力,特别适合于在军舰、科考船、海上钻井平台等起降条件和飞行环境较差的平台上执行任务。
舰载直升机能够有效扩大舰船作业半径、丰富海上作业科目、提升应对海上突发情况的能力。
发展舰载直升机对于保障舰船航行安全、维护海洋权益具有重要意义。
世界各海洋大国和航空大国历来重视舰载直升机的发展,相继研制出各种不同构型的舰载直升机(见表\ref{Marine-Helicopter})。
我国海军正处于“近海防御型向近海防御与远海护卫型结合转变”\upcite{PRC2015}的关键时期,发展适合中国国情的舰载直升机,对于实施海洋强国战略具有重要意义。
\begin{longtable}[c]{cccccl}
\caption{国外主要舰载直升机}
\label{Marine-Helicopter}\\\toprule[1pt]
原产国 & 型号 & 别名 & 构型 & 首飞时间 & 服役时间
\\\midrule
美国 & SH-60 & 海鹰(Sea Hawk) & 单主旋翼 & 1979 & 1984-现在  \\
美国 & CH-46 & 海骑士(Sea Knight) & 纵列式 & 1962 & 1964-2004\footnote{美国海军}/2015\footnote{美国海军陆战队}  \\
美国 & V-22 & 鱼鹰(Osprey) & 倾转旋翼 & 1989 & 2007-现在  \\
苏联 & Ка-27/28 & 蜗牛(Helix) & 单主旋翼 & 1973 & 1982-现在  \\
\bottomrule[1pt]
\end{longtable}

在舰载直升机的设计、使用和维护过程中,存在大量亟待解决的科学和工程问题。
其中,既有旋翼飞行器所共有的一般性问题,也有直升机执行海上作业任务所带来的特殊问题。

\section{旋翼飞行器综合分析}
旋翼(Rotor)是以直升机(Helicopter)为代表的旋翼飞行器(Rotorcraft)区别于其他航空飞行器(Aircraft/Aerocraft)的标志性部件。
从力学研究的角度来看,旋翼系统(以及旋翼-机体耦合系统)属于复杂的刚柔耦合多体动力学系统。
与其他复杂系统一样,对旋翼系统(以及旋翼-机体耦合系统)的研究大致可以分为实验和计算两大类。

实验研究,是人们对科学和工程问题所有理性认识的主要来源,也是检验一切物理模型和计算方法合理性的重要依据。
对旋翼飞行器的实验研究,
按实验对象可以分为缩比模型实验、全尺寸样机实验;
按实验环境可以分为风洞实验、试飞实验。
由于代价昂贵,只有少数大型研究机构和企业有能力对旋翼飞行器开展系统的实验研究。

计算研究,是对实验研究的重要补充,有时也被称为数值实验。
它是指从运动学、动力学、空气动力学等学科的基础理论出发,利用计算机,对直升机(特别是旋翼)的力学行为进行分析计算。
相对于模型或样机实验,数值实验拥有巨大的成本优势,这是驱动计算研究不断发展的主要动力。
对于新的设计方案,数值实验能够先于模型或样机实验对其性能指标进行评价,这也是其相对于模型或样机实验所具有的优势。
另外,对于飞行事故等模型或样机实验难以复现的过程,数值实验也是不可替代的研究手段。
可以预见,随着计算机硬件和分析软件水平的不断提高,数值实验将在直升机设计、使用和维护过程中扮演越来越重要的角色。

直升机界于1980年左右提出了“综合分析(Comprehensive Analysis)”的概念。
美国直升机界的权威专家、著名直升机综合分析软件CAMRAD\footnote{Comprehensive Analytical Model of Rotorcraft Aerodynamics and Dynamics}
的开发者Wayne Johnson将综合分析定义为:
“分析直升机在气动载荷作用下力学行为的通用计算程序”\upcite{Johnson2012}。
这里的“综合(Comprehensive)”应从以下几个角度来理解:
\begin{compactdesc}
  \item[分析内容]
  综合分析涵盖了与直升机相关的所有力学问题,包括:总体性能、配平、桨叶运动、气动载荷、结构载荷、振动、噪声、气弹稳定性、气弹响应、飞行品质等。
  \item[分析对象]
  综合分析适用于所有直升机构型(单旋翼带尾桨式、共轴式、纵列式、横列式)和旋翼构型(全铰式、跷跷板式、球铰式、无铰式、无轴承式)。
  \item[涉及学科]
  综合分析涉及空气动力学、结构动力学、飞行动力学等多个应用力学分支。
  \item[扩展性]
  综合分析的模型和程序应当具有扩展性,能够适用于新的构型设计方案,能够方便地引进新的物理模型和计算方法。
  \item[通用性]
  综合分析应当贯穿于直升机设计和改进过程的每个阶段。
\end{compactdesc}

要实现上述综合分析,必须首先解决空气动力学、动力学分析的若干关键技术。
过去几十年,这些学科各自独立的分析方法都得到了不同程度的发展,并且已经开发出许多通用的大型分析软件。
理想的综合分析是在现有的认识水平下,采用各学科最先进的分析技术,以达到尽可能高的计算精度。
然而,由于理论和分析技术的发展水平参差不齐,目前尚无法将各学科最先进技术的分析技术很好地整合在一起。
建立真正体现多物理场耦合的综合分析平台,始终是各国直升机界共同的奋斗目标。

\subsection{旋翼空气动力学}
在综合分析所涉及的几个主要学科中,空气动力学被认为是制约上述目标实现的主要瓶颈。
事实上,旋翼空气动力学也是流体力学研究的难点和热点,这主要是因为旋翼所处气流环境具有非定常、非线性的特点。

非定常主要表现在以下两个方面:
首先,由于旋翼始终相对机身运动,即使在悬停状态,旋翼-机身系统所处的气流环境也是随桨叶方位变化而变化的;
其次,随着旋翼高速旋转,桨叶当地迎角剧烈变化,局部会出现大迎角甚至反流状态,桨叶剖面的动态失速特性十分明显\upcite{Johnson1998Aero}。

非线性主要体现在以下两个方面:
其一,气动力与当地迎角、马赫数、雷诺数的关系,在较大的迎角、马赫数范围内没有线性的(甚至也没有解析的)表达式;
其二,旋翼尾迹由桨尖涡主导,机身、固定翼面尾迹与旋翼桨尖涡之间存在复杂的非线性自诱导和互诱导现象。

非定常气动力、动态失速、空气压缩性 、桨尖涡主导的旋翼尾迹分析构成了旋翼空气动力学的主要研究内容\upcite{Johnson1995Wake}。
研究方法大致分为实验和数值计算两大类。
实验研究包含定性流动显示实验、测力实验、测速实验、定量流动显示实验。
其中,基于三维粒子成像测速技术的高分辨率定量流动显示实验的是该领域的研究热点。
数值计算研究包含旋翼入流模型、涡线尾迹模型、黏性涡粒子方法、基于欧拉观点的涡量输运方程、计算流体动力学(Computational Fluid Dynamics, CFD)等方法。
其中,CFD方法又分为雷诺平均Navier-Stokes方程、大涡模拟、直接数值模拟等几个层次。
尽管直接数值模拟被认为是流体力学数值计算研究的终点,但旋翼流场飞行这一复杂流动问题的多尺度特性决定了该方法短时间内还难以应用于解决实际工程问题。
基于流动特征的混合方法、自适应网格加密技术、高性能并行计算等,是目前较为可行的替代手段。

\subsection{动力学}
动力学问题最基本、最一般的表述已由Hamilton原理给出。
%\begin{equation}
%\delta \int_{t_1}^{t_2} \left( T-V+W \right) \mathrm{d} t = 0
%\end{equation}
%其中,$T$为系统动能,$V$为保守力(重力、弹性力)的势,$W$为非保守力的功。
%
旋翼系统、旋翼-机体系统的动能、势能(包括重力势能和弹性势能)、能量耗散函数、外力功的表达式原则上都可以按部就班地推出,但具体实施起来相当繁琐。
针对某种特定构型,建立相应的动力学模型原则上并不困难。
如何使建模方法具有通用性和扩展性,则是该领域的研究热点。
为使推导过程尽可能具有通用性,尽可能减少手动推导的工作量,需引入多体系统动力学的思想和方法\upcite{HongJZ1999,Wittenburg2008,Shabana2013}。

早期的旋翼动力学分析采用小挥舞角、小摆振角、小变形假设,为的是得到线性或局部线性的表达式,以便于通过解析的方法进行分析计算。
但这样得出的动力学关系式并不能真实反映旋翼系统的动力学特性。
进入21世纪,计算机性能相对于几十年前已经获得了很大程度的提高,以往被认为无法承受的计算量,现在已经能够在高性能工作站或集群上进行处理。
另一方面,整个综合分析的计算量主要集中在气动分析方面,如果动力学模块的算法设计得合理,考虑几何非线性一般不会明显增加计算量。
但几何非线性给动力学分析带来了新的问题,例如结构动力学分析中常用的有限元方法(Finite Element Method, FEM),在处理大变形问题时会出现网格畸变,从而影响计算效果和计算效率。

动力学方程的规范列式和相应的数值求解方法,以及对非线性耦合效应的处理,是综合分析在动力学方面需要解决的关键问题\upcite{Johnson1998Dyna}。

\section{舰载直升机}
以上所列举的旋翼飞行器所共有的一般性问题,是研究舰载直升机空气动力学、结构动力学、飞行动力学问题的基础。
海面、舰面特有的气流、动力学环境,则为舰载直升机设计、使用及维护带来了一些新的问题。

舰船在海面航行时,船体结构对海面气流的扰动使得直升机起降平台附近的气流环境较为复杂,增加了直升机完成起降飞行任务的难度。
海面的风速和风向、舰船的航速和航向都是影响上述船体空气尾流(Ship Airwake)的重要因素。

海浪和海风引起的船体起伏、俯仰、滚转运动,使舰载直升机舰面动力学问题的复杂程度,大大高于直升机在地面和空中飞行时所对应的动力学问题。
直升机旋翼尾迹与船体空气尾流之间存在复杂的相互作用,进一步增加了上述问题的复杂程度。

建立合适的海风模型、船体空气尾流模型、气动干扰模型,是舰载直升机空气动力学研究的主要任务;
研究舰载直升机在各种海浪和海风条件下的动力学稳定性,是舰载直升机结构动力学研究的主要内容;
确定舰载直升机在各种海风条件和舰船航行状态下的安全飞行路径和操纵策略,则是舰载直升机飞行动力学、导航及控制方法研究的重点和难点。

\section{小结}

在舰载直升机设计和使用所面临的各种科学和工程问题中,空气动力学是其中最核心、最关键的问题。
国内外众多学者在这方面开展了大量研究工作,积累了丰富的研究方法和研究成果。
本文对旋翼空气动力学、舰船空气动力学以及计算空气动力学新方法的研究现状进行了综述,介绍了与舰载直升机相关的一些应用问题,并总结了该领域在未来一段时间内的发展趋势。
