% !Mode:: "TeX:UTF-8"
\chapter{结论}
随着我国海洋强国战略的逐步实施,我国在海上的军事、科考和生产活动正变得日益频繁。
与之相伴的是海上作业任务的不断丰富,以及对舰船海上作业半径和反应时间要求的不断提高。
飞行器以其远高于舰船的运动速度,能够极大地延伸舰船作业半径,缩短反应时间。
固定翼飞机对起降平台的要求较高,通常只有在具有数百米长跑道的航母上才能完成。
直升机以其特有的垂直起降能力,可以在与其尺度大致相当的平台上完成起降,因而比固定翼飞机更广泛地应用于海上作业。

但是,海面复杂多变的气流和海浪条件,船体对海面气流的非定常扰动,海面、船体与旋翼之间的气动干扰,
使得直升机在舰面完成起降任务的难度远大于其在陆地上执行起降任务,也大于固定翼飞机在航母平台上执行舰面起降任务。
着舰和离舰是舰载直升机执行任何海上作业任务都必须完成的飞行科目。
为确保舰载直升机安全完成舰面起降任务、减轻飞行员工作负担,有必要对舰载直升机的空气动力学、舰面动力学、飞行动力学、导航和飞行控制方法开展深入研究。
其中,空气动力学是后续几项研究内容的基础。
本文对舰载直升机空气动力学的国内外研究现状进行了综述,并对与舰载直升机相关的一些应用问题作了简要介绍。

舰载直升机空气动力学研究,既包含所有旋翼飞行器共有的一般性问题(即通常意义下,旋翼空气动力学所包含的研究内容),也有海面和船体与旋翼相互影响所带来的特殊问题。
旋翼空气动力学研究的主要任务是,认识旋翼流场的特点和规律,为旋翼飞行器的设计、使用和维护提供支持。
舰载直升机舰面空气动力学研究的主要任务则是,深入理解旋翼尾迹和船体空气尾流的流动特点,
把握海面、船体与旋翼之间的气动干扰规律,为舰载直升机舰面动力学、飞行动力学、导航和飞行控制方法研究提供支持。
这两大类问题的研究方法均可分为实验和计算两类。

对旋翼空气动力学的实验研究经历了从定性到定量,从宏观到微观的发展过程。
最开始,人们通过自然凝结现象,观察到了旋翼桨尖涡的存在,并认识到旋翼流场由桨尖涡主导的重要事实。
此后,喷烟法、阴影法、纹影法等流动显示方法不断被引入旋翼空气动力学的定性实验研究中。
定量实验方面,包含测力实验和测速实验两类。
伴随实验技术的进步,测速实验经历了侵入式热线测速技术、非侵入式单点LDV技术、非侵入式多点PIV技术的发展过程。
目前,三维PIV技术已经能够对旋翼流场进行高分辨率测量,可以预见,该技术在未来一段时间里仍将是旋翼空气动力学研究的重要工具。

数值计算是旋翼空气动力学的另一类重要的研究方法。
旋翼流场的非定常、非线性特征决定了问题的复杂性。
为了真实还原旋翼流场的流动特征,必须解决好转捩、附面层分离、涡核粘性耗散、涡结构失稳破裂等复杂的流体力学问题。
在连续介质力学框架内,只有直接数值模拟技术可以完全实现上述目标。
受限于计算机硬件和软件发展水平,将DNS技术直接应用到旋翼空气动力学计算上,目前看来还很不现实。
因此,有必要寻找切实可行的替代方法。
旋翼入流模型、涡线尾迹模型、黏性涡粒子模型、基于欧拉观点的涡量输运模型、雷诺平均Navier-Stokes方程模型等,都已经被直升机工程界所采用。
为了解决网格离散带来的涡量非物理耗散问题,网格自适应加密技术、涡量约束方法等新兴的流体力学计算方法正成为该领域的研究热点。
而无网格法以其处理复杂几何边界的灵活性,也逐渐引起了计算流体力学界的关注。

除上述旋翼空气动力学的一般性问题外,舰载直升机还有一些海上作业、舰面起降所带来的特殊空气动力学问题。
这类特殊问题主要表现为海面自由来流、舰船空气尾流与旋翼尾迹的相互作用,即气动干扰问题。
美国、法国、荷兰等国在该领域开展了许多实验和计算方法研究,一些分析方法已经实现商业化,并成功应用到飞行仿真、导航和飞行控制系统设计等应用领域。
相比之下,我国在这方面开展的研究工作十分有限。
研究舰载直升机空气动力学,特别是海面、舰船与旋翼之间的复杂气动干扰问题,
对于确定直升机舰面起降的环境条件,制定和完善直升机舰面起降作业规程,提高舰载直升机的安全性和作业效率具有重要意义。
