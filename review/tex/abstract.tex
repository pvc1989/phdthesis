% !Mode:: "TeX:UTF-8"
\begin{cabstract}
直升机特有的垂直起降和悬停飞行能力,特别适合于在海上平台执行作业任务。
在所有与舰载直升机相关的力学问题中,空气动力学是最核心、最关键的问题。
对舰载直升机而言,空气动力学的研究内容主要包含两部分:
一是所有旋翼飞行器所共有的一般性问题,二是舰船所带来的特殊问题。

旋翼流场由桨尖涡主导,具有非定常、非线性的特点,复杂程度高于固定翼流场。
对旋翼空气动力学的研究主要分为实验和数值计算两大类。
实验研究经历了从定性到定量、从宏观力学参数测量到全流场流动细节捕捉的发展过程。
数值计算方面,相继发展出了入流模型、涡线尾迹模型、黏性涡粒子模型、基于欧拉观点的涡量输运方程模型、雷诺平均Navier-Stokes方程模型等分析方法。
由于旋翼空气动力学本身的复杂性和现有研究手段的局限性,该领域目前仍是空气动力学研究的热点和难点。

海面自由来流经过舰船表面结构,形成紊乱的空气尾流,与旋翼尾迹的相互作用,形成舰载直升机所特有的空气动力学问题。
舰载直升机处于舰面起降过程中时,海面和舰面会对旋翼产生“地”面效应,这也是海面、舰面与旋翼之间复杂气动干扰的一种形式。
把握上述气动干扰的规律,是舰载直升机空气动力学的主要研究内容。
与旋翼空气动力学类似,该领域的研究方法也分为实验和计算两大类。

本文对舰载直升机空气动力学的国内外研究现状进行了综述,并对与舰载直升机相关的一些应用问题作了简要介绍。
\end{cabstract}
%
\begin{eabstract}
Helicopter is particularly suitable for offshore platforms, for its unique ability of vertical taking off and landing and hovering.
Among all the problems of mechanics on a ship borne helicopter, aerodynamics is the most important and key one.
It mainly contains two kinds of problems, one is about the common problems of rotorcraft aerodynamics, the other is the specific issues brought by the ship.

The flow field of a rotor is dominated by the tip vortices.
It is much more sophisticated than that of a fixed wing, since it is highly unsteady and nonlinear.
There are two major categories of research approaches: experiment and numerical calculation.
Experimental studies have experienced great development from qualitative to quantitative, from measuring macro mechanical parameters to capturing details of the entire flow field.
In the aspect of numerical calculation studies, different kinds of calculation models have been developed, such as rotor inflow models, vortex line wake models, the viscous vortex particle model, 
the Eulerian vorticity transport equation model, the Reynolds averaged Navier Stokes equation model, etc..
Due to the complexity of the rotor aerodynamics and the limitations of existing research methods, this field will continue to be a hot and difficult spot in the school of aerodynamics.

As free winds pass through the building structures on the deck of a ship, chaotic ship airwakes are formed at the same time.
When a helicopter is performing a taking off or landing task, there are strong and complex interactions between the wake of its rotor(s) and the ship airwakes.
Ground effects can also be experienced by the rotor(s) during this task, because of the existence of the sea surface and the ship deck.
It is the most important task for ship borne helicopter aerodynamics to understand such sea-ship-helicopter aerodynamic interactions. 
The research methods in this field, like those in rotor aerodynamics, is also divided into two categories: experimental and computational.

In this paper, the research status of rotor aerodynamics and ship borne helicopter aerodynamics, both in China and abroad, has been reviewed.
Some problems related to the application of the ship borne helicopters are briefly introduced.
\end{eabstract}
