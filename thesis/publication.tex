\chapter{攻读博士学位期间取得的学术成果}

\begin{enumerate}
\item
Pei W, Jiang Y, Li S. An Efficient Parallel Implementation of the Runge–Kutta Discontinuous Galerkin Method with Weighted Essentially Non-Oscillatory Limiters on Three-Dimensional Unstructured Meshes. \textit{Applied Sciences}, 2022, 12(9): 4228.

\item
Pei W, Jiang Y, Li S. High-Order CFD Solvers on Three-Dimensional Unstructured Meshes: Parallel Implementation of RKDG Method with WENO Limiter and Momentum Sources. \textit{Aerospace}, 2022, 9(7): 372.

\item
裴为诚, 刘畅, 蔡玉洁, 李书. 舰载直升机空气动力学及其应用现状. \hwkaiti{航空科学技术}, 2022, 33(10): 1-15.

\item
Pei W, Li S. Computational Method for the Aeromechanical Behaviors of a Rotorcraft in Vortex Ring State. \textit{ICCES (International Conference on Computational \& Experimental Engineering and Sciences)}, July 20-24, 2015, Reno, NV, United States.

\item
Wang X, Pei W, S N Atluri. 
Bifurcation \& chaos in nonlinear structural dynamics: Novel \& highly efficient optimal-feedback accelerated Picard iteration algorithms. \textit{Communications in Nonlinear Science and Numerical Simulation}, 2018, 65: 54-69.
\end{enumerate}

