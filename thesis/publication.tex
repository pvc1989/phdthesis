\chapter{攻读博士学位期间取得的学术成果}
\paragraph*{用于申请学位的学术论文:}
\begin{enumerate}
\item
\emph{Pei Weicheng}, Jiang Yuyan and Li Shu. An Efficient Parallel Implementation of the Runge–Kutta Discontinuous Galerkin Method with Weighted Essentially Non-Oscillatory Limiters on Three-Dimensional Unstructured Meshes. \textit{Applied Sciences}, 2022, 12(9): 4228.
DOI: \href{https://doi.org/10.3390/app12094228}{10.3390/app12094228}

\item
\emph{Pei Weicheng}, Jiang Yuyan and Li Shu. High-Order CFD Solvers on Three-Dimensional Unstructured Meshes: Parallel Implementation of RKDG Method with WENO Limiter and Momentum Sources. \textit{Aerospace}, 2022, 9(7): 372.
DOI: \href{https://doi.org/10.3390/aerospace9070372}{10.3390/aerospace9070372}.

\item
\emph{裴为诚}, 刘畅, 蔡玉洁, 李书. 舰载直升机空气动力学及其应用现状. \hwkaiti{航空科学技术}, 2022, 33(10): 1-15.
DOI: \href{https://doi.org/10.19452/j.issn1007-5453.2022.10.001}{10.19452/j.issn1007-5453.2022.10.001}
\end{enumerate}

\paragraph*{其他成果:}
\begin{enumerate}
\item
\emph{Pei Weicheng} and Li Shu. Computational Method for the Aeromechanical Behaviors of a Rotorcraft in Vortex Ring State. \textit{ICCES (International Conference on Computational \& Experimental Engineering and Sciences)}, July 20-24, 2015, Reno, NV, USA.

\item
Xuechuan Wang, \emph{Weicheng Pei} and Satya N. Atluri. 
Bifurcation \& chaos in nonlinear structural dynamics: Novel \& highly efficient optimal-feedback accelerated Picard iteration algorithms. \textit{Communications in Nonlinear Science and Numerical Simulation}, 2018, 65: 54-69.
DOI: \href{https://doi.org/10.1016/j.cnsns.2018.05.008}{10.1016/j.cnsns.2018.05.008}
\end{enumerate}

