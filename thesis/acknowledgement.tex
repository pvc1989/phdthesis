\chapter{致谢}

经过近十年的跋涉,我的研究生阶段终于到了可以做全面总结的时候。
值此博士学位论文提交之际,请允许我向这十年间关心和帮助过我的师长、同门、校友表示衷心的感谢。

首先要感谢我的导师李书教授。
他在学习、科研和生活上给予我的充分信任、全面支持和亲切关怀,是我能坚持走到读研生涯这最后一步的最重要的外部支持。
李老师不仅是我这十年学业上的导师,也将是我今后几十年工作和生活上的导师。

我在研究生阶段的学习和研究,主要包括直升机空气动力学和计算力学两大领域。
\begin{itemize}[wide]
\item
在直升机气动领域,贺天鹏老师是我最早的引路人。是他的指导和鼓励,帮助我快速进入这一充满挑战的领域,并逐渐寻找到适合自己的研究方向。
在直升机相关的科研工作中,简成文、尚红星、安强林、朱文国等诸位同门与我开展了亲密无间的合作,是我科研道路上最早的一批战友。
尚红星、安强林二位师弟还在学习和工作之余,协助我完成了实验室机群的采购,为本文的研究工作积累了必要的硬件基础。
\item
在计算力学领域,Atluri 院士和董雷霆教授是我最重要的向导。
他们扎实的理论功底、宽广的知识储备与开阔的学术视野让我受益匪浅。
张韬、范祺锋二位师兄的出国访学经历,为我的学术生涯规划提供了重要的参考。
在访学期间,汪雪川、王冠楠、Maryam 三位博士以及 Sladek 教授,都与我开展了不同程度的学术交流。
汪雪川、王冠楠二位博士还在生活上给予了我极大的便利,是我在异国他乡最亲最近的同胞和伙伴。
\end{itemize}

本文的研究工作,是对以上两个领域学习成果的融会贯通和综合应用。
以下三位同门分别在此项研究工作的不同阶段,直接参与了一对一合作。
\begin{itemize}
\item 李亦民师弟于 2016 年暑假,与我一起历经多次失败,最终成功搭建了并行计算环境,为本文的研究工作提供了最主要的软件平台。
\item 杨明浩师弟自 2018 年 9 月起,与我一同系统地学习了现代 CFD 方法,并于 2019 年暑假与我一同实现了二维有限体积求解器。
\item 江毓彦师妹自 2020 年 9 月起,承担了非结构网格分块、基函数正交归一化等模块的编码工作,与我一起实现了并行化的三维间断有限元求解器。
\end{itemize}
以上三位同门是我在博士论文研究工作中最主要的助手。在此指明其贡献,以示感谢。

最后,要感谢 D520 的所有同门,以及北京航空航天大学、浙江大学的诸位校友。
是你们陪伴我度过了人生最宝贵的十年。

时有师者勤源赠诗一首云:
\begin{quote}
% 壬寅七月癸巳,梅公业竟,诸人致贺。余自受任从属,朝乾夕惕,相与交游,及今凡十有一年。梅公尝从容曰:“此间乐,不思蜀也。”余问之,则叹曰:“乃心西悲,无日不思。”余哂之,转而戚戚,故作。
\centering{}
沙洲败世有孤芳。独向荆榛启秽荒。\\
定鼎非求方夏禹,诛凶未欲比商汤。\\
临行叹尽乌衣重,再会嗟余夜路长。\\
客座梁台悲蜀舞,相期并力整遗棠。\\
\end{quote}

