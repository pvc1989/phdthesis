% !Mode:: "TeX:UTF-8"
\chapter*{结论\markboth{结论}{}}
\addcontentsline{toc}{chapter}{结论}

本文首先对旋翼空气动力学、舰载直升机气动干扰、高精度计算流体动力学 (CFD) 方法的国内外研究现状进行了文献调研。在此基础上,以舰载直升机气动干扰的高效高精度数值模拟为目标,提出了一种既能处理船体复杂几何外形、又能高效模拟旋翼对流场扰动的“动量源--有限元”混合方法,并给出了该方法的一种支持分布式并行计算的代码实现。

\subparagraph{本文的主要结论如下:}
\begin{enumerate}[wide]
\item 适用于三维非结构网格的\uline{龙格--库塔间断伽辽金 (RKDG)} 求解器,能够处理具有复杂几何外形的边界,从而适用于航空航天领域的工程问题。
\item 基于邻接单元的\uline{加权基本无振荡 (WENO)} 重构过程,能够有效地压制间断附近的数值振荡,同时保持 RKDG 方法的紧致性、可并行性。
\item 高阶 RKDG 求解器的数值耗散较低,能够在相对较粗的非结构网格上捕获复杂的流动细节,具有更高的性价比。
\item 非定常动量源模型与高阶 RKDG 求解器配合使用,能够在静态网格上高效、高精度地模拟旋翼对流场的扰动,避免了传统旋翼空气动力学数值模拟方法对动态网格、滑动网格、重叠网格等高级网格技术的依赖。
\end{enumerate}

\subparagraph{本文的主要创新点如下:}
\begin{enumerate}[wide]
\item 基于现代 C++ 的标准库、类机制、模板机制,以类模板的形式给出了一种 RKDG 方法的通用代码实现。该模板通过解除空间离散、时间推进、数值通量之间的相互依赖,允许这些模块各自独立开发、任意组合,提高了代码复用率、可维护性和可扩展性。
\item 提出并实现了一种高效、高精度的旋翼空气动力学模拟方法。该方法在上述通用 RKDG 求解器中,嵌入了一种\uline{非定常动量源
(UMS)} 模型,用于模拟旋翼对流场的扰动。所得的 UMS--RKDG 求解器被成功地应用于旋翼数值风洞试验与舰载直升机气动干扰问题的研究。
\item 给出了上述通用 RKDG 求解器及其旋翼适配版 UMS--RKDG 求解器的并行化实现,使其能够更充分地利用现有计算资源来加速程序的运行,从而缩短数值模拟在实际应用中的迭代周期。
\end{enumerate}

\subparagraph{以下几点可以作为进一步研究的方向:}
\begin{enumerate}[wide]
\item 支持处理高阶导数项,如耗散项、黏性项。对线性单波而言,这意味着用线性对流--扩散方程,替换本文中求解的线性平流方程;对流体力学而言,这意味着用纳维--斯托克斯方程,替换本文中求解的欧拉方程。一种可能的方案是采用\uline{局部间断伽辽金
(LDG)} 方法,将高阶偏微分方程化为一阶偏微分方程组,再为新引入的未知量设计数值通量,从而使 RKDG 方法的数据结构本地化、高度可并行化等优点得以保持。
\item 提高前/后处理模块的并行化程度。本文只实现了求解器模块的并行化,前处理模块中的网格生成及分块功能、后处理模块中的数据可视化功能实际上都是通过调用第三方串行程序完成的。这种串行环节的存在,构成了限制问题规模的瓶颈,降低了整个程序的并行效率,浪费了并行计算机的硬件资源。
\item 在公共高性能计算平台上测试并行性能。本文中的所有并行算例都是在个人计算机或实验室自建小型机群上完成的,通信网络的拓扑结构相对简单,实际运行中没有出现拥塞或连接故障,测得的并行性能可能偏乐观。更大规模的计算机群必然意味着更加复杂的通信网络,出现通信故障的概率也随之增大,通信与计算的重叠可能更难实现。
\end{enumerate}
