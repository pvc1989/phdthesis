
\chapter*{结论与展望}

\addcontentsline{toc}{chapter}{结论与展望}
\chaptermark{结论与展望}

本文首先对旋翼空气动力学、舰载直升机气动干扰、高精度 CFD 方法的国内外研究现状进行了文献调研。在此基础上,作者以舰载直升机气动干扰的高效高精度数值模拟为目标,提出了一种既能处理船体复杂几何外形、又能高效模拟旋翼对流场扰动的
UMS–RKDG 混合方法,给出了该方法的一种支持分布式并行计算的 C/C++ 代码实现,并在实验室自建的小型机群上测试了并行性能。

\section*{本文的主要结论}
\begin{enumerate}[wide]
\item 适用于三维非结构网格的 RKDG 求解器,能够处理具有复杂几何外形的边界,适用于航空航天领域的工程问题。
这些问题通常以具有复杂外形的三维几何模型为输入,其复杂性主要来源于普遍存在的各种三维曲面,以及不规则的拓扑连接关系。
为这种几何模型生成结构网格,即便可行也需耗费大量人力资源;而现有的自动化网格生成工具,虽然能够极大地减少人为干预,但其输出的网格一般而言都是非结构的。
这种网格生成层面的矛盾,对求解器的影响主要体现在空间离散方法的选择上:基于有限差分法的求解器只能应用于结构网格,而基于有限体积法或有限单元法的求解器则不受此限制。
本文选择实现基于有限单元法的 RKDG 求解器,正是为了使其适用于能够自动生成的三维非结构网格。

\item 基于邻接单元的 WENO 重构过程,能够有效地压制间断附近的数值振荡,同时保持 RKDG 方法的紧致性、可并行性。
航空航天领域的工程问题存在大量高速气流,后者因压缩性明显而普遍存在各种间断(激波、膨胀波、接触间断)。
间断的存在对高阶格式的设计提出了额外的挑战:间断函数在有限维连续函数(例如:多项式)空间内的 $L_2$ 投影存在数值振荡,必须加以抑制(例如:滤波、限制、人工耗散)才能避免出现非物理解(例如:负密度、负压强)。
在所有数值振荡抑制方法中,WENO 重构过程(限制器)具有较为出众的性能:它既能有效地压制非物理的数值振荡、又能避免在光滑极值点处降低精度。
除此之外,限制器的构造还应保持其所依托的求解器既有的优势。本文选择实现基于邻接单元的 WENO 限制器,正是为了保持 RKDG 求解器的紧致性、可并行性。

\item 高阶 RKDG 求解器的数值耗散较低,能够在相对较粗的非结构网格上捕获复杂的流动细节,具有更高的性价比。
数值耗散是评价求解器(及其底层离散格式)品质的重要指标,通常认为某种求解器的数值耗散越低则该求解器的品质越好。
本文实现的高阶 RKDG 格式,在所有算例中都表现出低于(普遍应用于商业软件的)低阶格式的数值耗散,这一点在粗糙的非结构网格上体现得尤为明显。
在精度上取得上述收益的代价是程序开发难度的提升以及运行时间和存储空间的增长。
本文通过比较不同求解器–网格 ($p$–$h$) 组合给出的计算结果和资源开销,得出了“RK3/DG3 求解器相对于 RK1/DG1 求解器具有更高性价比”的结论。

\item 非定常动量源 (UMS) 模型与高阶 RKDG 求解器配合使用,能够在静态网格上高效、高精度地模拟旋翼对流场的扰动,避免了传统旋翼 CFD 方法对动态网格、滑动网格、重叠网格等高级网格技术的依赖。
这些高级网格技术普遍依赖于三维几何查找,后者在结构网格或具有层次结构的 Cartesian 网格上比较容易实现,但在一般的非结构网格上实现起来难度较大(即使有理论实现方案,其实际计算效率也不够高)。
本文选用动量源模型来模拟桨叶对流场的扰动,正是为了实现在静态网格上完成旋翼三维流场计算。
将代表桨叶气动力的动量源模型植入 CFD 求解器的实践早已有之,但鲜有与高阶 CFD 求解器结合的案例,而 UMS–RKDG 这种组合则属本文首创。

\item 动量源附近存在物理间断,可能在高阶近似解中引起数值振荡,必须像处理高速气流中的间断那样加以抑制。
尽管直升机的飞行速度属于低亚声速,正常情况下不会出现激波,但考虑到桨尖局部的相对速度有可能跨声速,求解器还是应该具备激波捕捉和抑制振荡的能力。
除此之外,本文提出的 UMS–RKDG 方法在代表桨叶的动量源附近引入了压强间断,这种间断(与激波、接触间断类似)对高阶 RKDG 格式亦有压制数值振荡的要求。
本文实现的 WENO 限制器,能够无差别地处理上述两种不同来源的间断,从而最大限度地保持了高阶 RKDG 求解器紧致、易并行的优势。
\end{enumerate}


数值试验表明:本文提出的 UMS–RKDG 求解器是模拟旋翼流场、研究舰载直升机气动干扰的一种高效、高精度的计算方法。


\section*{本文的主要创新点}
\begin{enumerate}[wide]
\item 提出、实现并验证了一种适合具有复杂边界(例如:直升机机身、船体结构)流场计算的高阶 (RKDG) 有限元方法。
该方法能够在(自动生成的)三维非结构网格上构造统一的高阶格式,支持分布式并行计算,具有良好的可扩展性,能够应用于航空航天领域的实际工程问题。
具有解析解的一维标准算例和具有高精度数值解的二维标准算例,都验证了该方法的正确性、高性价比和可并行性。

\item 提出、实现并验证了一种适合旋翼流场计算的动量源–高阶有限元混合方法。
该方法将一种表示旋翼气动力的非定常动量源 (UMS) 模型嵌入到本文实现的 (RKDG) 有限元求解器中,既保持了高阶有限元格式紧致型、可并行性,又避免了传统旋翼 CFD 方法所依赖的动态网格技术带来的计算量剧增。

\item 提出、实现并验证了一种间断处理方法,将三维紧致 (WENO) 限制器用于抑制高阶 (RKDG) 有限元近似解在动量源附近的数值振荡。
指出在动量源–高阶有限元混合方法中引入限制器的必要性,并给出在 UMS–RKDG 求解器中引入 WENO 限制器的实例,是本文最主要的贡献。
\end{enumerate}


\section*{进一步研究的方向}
\begin{enumerate}[wide]
\item 支持处理高阶导数项,如耗散项、黏性项。对线性单波而言,这意味着用线性对流–扩散方程,替换本文中求解的线性平流方程;对流体力学而言,这意味着用
Navier–Stokes 方程,替换本文中求解的 Euler 方程。 
\item 提高前/后处理模块的并行化程度。本文只实现了求解器模块的并行化,前处理模块中的网格生成及分块功能、后处理模块中的数据可视化功能实际上都是通过调用第三方串行程序完成的。这种串行环节的存在,构成了限制问题规模的瓶颈,降低了整个程序的并行效率,浪费了并行计算机的硬件资源。
\item 在公共高性能计算平台上测试并行性能。本文中的所有并行算例都是在个人计算机或实验室自建小型机群上完成的,通信网络的拓扑结构相对简单,实际运行中没有出现拥塞或连接故障,测得的并行性能可能偏乐观。更大规模的计算机群必然意味着更加复杂的通信网络,出现通信故障的概率也随之增大,通信与计算的重叠可能更难实现。
\end{enumerate}
