% !Mode:: "TeX:UTF-8"
\chapter*{结论与展望\markboth{结论与展望}{}}
\addcontentsline{toc}{chapter}{结论与展望}

本文首先对旋翼空气动力学、舰载直升机气动干扰、高精度 CFD 方法的国内外研究现状进行了文献调研。在此基础上,以舰载直升机气动干扰的高效高精度数值模拟为目标,提出了一种既能处理船体复杂几何外形、又能高效模拟旋翼对流场扰动的 UMS--RKDG 混合方法,并给出了该方法的一种支持分布式并行计算的代码实现。

\subparagraph{本文的主要创新点如下:}
\begin{enumerate}[wide]
\item 提出并验证了一种适合舰载直升机流场计算的三维并行间断有限元方法。该方法适用于三维非结构网格、支持分布式并行计算,标准算例验证了该方法的有效性。
\item 提出并验证了一种适合旋翼流场计算的动量源–有限元混合方法。该方法将一种非定常动量源模型嵌入到上述三维并行间断有限元方法中,能够在三维非结构网格上构造高阶格式,同时避免动态网格技术带来的计算量剧增。
\item 提出并验证了一种三维紧致加权基本无振荡限制器,用于解决上述高阶间断有限元格式在旋翼桨叶等物理量间断处的数值振荡问题。
\end{enumerate}

\subparagraph{以下几点可以作为进一步研究的方向:}
\begin{enumerate}[wide]
\item 支持处理高阶导数项,如耗散项、黏性项。对线性单波而言,这意味着用线性对流--扩散方程,替换本文中求解的线性平流方程;对流体力学而言,这意味着用纳维--斯托克斯方程,替换本文中求解的欧拉方程。%一种可能的方案是采用\uline{局部间断伽辽金 (LDG)} 方法,将高阶偏微分方程化为一阶偏微分方程组,再为新引入的未知量设计数值通量,从而使 RKDG 方法的数据结构本地化、高度可并行化等优点得以保持。
\item 提高前/后处理模块的并行化程度。本文只实现了求解器模块的并行化,前处理模块中的网格生成及分块功能、后处理模块中的数据可视化功能实际上都是通过调用第三方串行程序完成的。这种串行环节的存在,构成了限制问题规模的瓶颈,降低了整个程序的并行效率,浪费了并行计算机的硬件资源。
\item 在公共高性能计算平台上测试并行性能。本文中的所有并行算例都是在个人计算机或实验室自建小型机群上完成的,通信网络的拓扑结构相对简单,实际运行中没有出现拥塞或连接故障,测得的并行性能可能偏乐观。更大规模的计算机群必然意味着更加复杂的通信网络,出现通信故障的概率也随之增大,通信与计算的重叠可能更难实现。
\end{enumerate}
