
\chapter{本文所用的数值积分器\label{chap:gaussian_pairs}}

\section{区间上的数值积分器}

一维单元总是可以借助等参变换化为标准区间 $(-1,1)$,从而可以用标准高斯求积公式构造数值积分器。标准高斯求积公式可以在数值分析教程(如文献 \cite{Quarteroni_2007})中找到,此处从略。

四边形单元、六面体单元总是可以借助等参变换化为相应维度的标准区间。这两种单元上的数值积分器可以由一维数值积分器的直积构造而得,无需使用
\ref{subsec:quadrature} 小节中介绍的通用方法。

\section{三角形单元上的数值积分器}

为了使用 \ref{subsec:quadrature} 小节中介绍的通用方法,需要解析地算出标准三角形上的积分
\begin{equation}
\int_{0}^{1}\int_{0}^{1-\eta_{B}}\eta_{A}^{a}\,\eta_{B}^{b}\dd{\eta_{A}}\dd{\eta_{B}}.
\end{equation}
此处证明一个更一般的结论:
\begin{equation}
I(a,b,c)=\int_{0}^{1}\int_{0}^{1-\eta_{B}}\eta_{A}^{a}\,\eta_{B}^{b}\,\eta_{C}^{c}\dd{\eta_{A}}\dd{\eta_{B}}=\frac{a!\,b!\,c!}{(a+b+c+2)!}.
\end{equation}
其中 $\eta_{A},\eta_{B},\eta_{C}$ 依次为 $A,B,C$ 三点对应的面积坐标。
\begin{proof}
将 $\eta_{C}=1-\eta_{A}-\eta_{B}$ 代入 $I(a,b,c)$,可得
\begin{equation}
\begin{aligned}I(a,b,c) & =\int_{0}^{1}\int_{0}^{1-\eta_{B}}\eta_{A}^{a}\,\eta_{B}^{b}\,\eta_{C}^{c}\dd{\eta_{A}}\dd{\eta_{B}}\\
 & =\int_{0}^{1}\left(\int_{0}^{1-\eta_{B}}\eta_{A}^{a}\left(1-\eta_{A}-\eta_{B}\right)^{c}\dd{\eta_{A}}\right)\eta_{B}^{b}\dd{\eta_{B}}\\
 & =\int_{0}^{1}\left(\int_{0}^{1}\left(\frac{\eta_{A}}{1-\eta_{B}}\right)^{a}\left(1-\frac{\eta_{A}}{1-\eta_{B}}\right)^{c}\frac{\dd{\eta_{A}}}{1-\eta_{B}}\right)\left(1-\eta_{B}\right)^{a+c+1}\eta_{B}^{b}\dd{\eta_{B}}
\end{aligned}
\end{equation}
利用 Beta 函数与 Gamma 函数的关系
\begin{equation}
\mathrm{B}(m+1,n+1)=\int_{0}^{1}t^{m}\,(1-t)^{n}\dd{t}=\frac{\Gamma(m+1)\,\Gamma(n+1)}{\Gamma(m+n+2)},
\end{equation}
及 Gamma 函数在正整数上的取值
\begin{equation}
\Gamma(n+1)=n!,\qquad\forall n\in\mathbb{N},
\end{equation}
可得
\begin{equation}
\begin{aligned}I(a,b,c) & =\int_{0}^{1}\left(\int_{0}^{1}\left(\frac{\eta_{A}}{1-\eta_{B}}\right)^{a}\left(1-\frac{\eta_{A}}{1-\eta_{B}}\right)^{c}\frac{\dd{\eta_{A}}}{1-\eta_{B}}\right)\left(1-\eta_{B}\right)^{a+c+1}\eta_{B}^{b}\dd{\eta_{B}}\\
 & =\frac{a!\,c!}{(a+c+1)!}\int_{0}^{1}\left(1-\eta_{B}\right)^{a+c+1}\eta_{B}^{b}\dd{\eta_{B}}\\
 & =\frac{a!\,c!}{(a+c+1)!}\frac{b!\,(a+c+1)!}{(a+b+c+2)!}=\frac{a!\,b!\,c!}{(a+b+c+2)!}
\end{aligned}
\end{equation}
\end{proof}
%

\subsection{一点一阶积分器}

该积分器只含 1 个取值点,其面积坐标为
\begin{equation}
\begin{bmatrix}a & a & a\end{bmatrix},\qquad a=\frac{1}{3},
\end{equation}
对应的权重为
\begin{equation}
w=\frac{1}{2}.
\end{equation}


\subsection{三点二阶积分器}

该积分器含 3 个取值点,其面积坐标为
\begin{equation}
\begin{bmatrix}a & a & b\end{bmatrix},\qquad\begin{bmatrix}a & b & a\end{bmatrix},\qquad\begin{bmatrix}b & a & a\end{bmatrix},
\end{equation}
其中
\begin{equation}
a=\frac{1}{6},\qquad b=1-2a,
\end{equation}
对应的权重为
\begin{equation}
w=\frac{1}{6}.
\end{equation}


\subsection{六点四阶积分器}

该积分器含 6 个取值点:
\begin{itemize}[wide]
\item 前 3 个点的面积坐标为
\begin{equation}
\begin{bmatrix}a & a & b\end{bmatrix},\qquad\begin{bmatrix}a & b & a\end{bmatrix},\qquad\begin{bmatrix}b & a & a\end{bmatrix},
\end{equation}
其中
\begin{equation}
a=0.44594849091596488631832925388305199,\qquad b=1-2a,
\end{equation}
对应的权重为
\begin{equation}
w=\frac{0.22338158967801146569500700843312280}{2}.
\end{equation}
\item 后 3 个点的面积坐标为
\begin{equation}
\begin{bmatrix}a & a & b\end{bmatrix},\qquad\begin{bmatrix}a & b & a\end{bmatrix},\qquad\begin{bmatrix}b & a & a\end{bmatrix},
\end{equation}
其中
\begin{equation}
a=0.09157621350977074345957146340220151,\qquad b=1-2a,
\end{equation}
对应的权重为
\begin{equation}
w=\frac{0.10995174365532186763832632490021053}{2}.
\end{equation}
\end{itemize}
%

\subsection{十二点六阶积分器}

该积分器含 12 个取值点:
\begin{itemize}[wide]
\item 第 1–3 号点的面积坐标为
\begin{equation}
\begin{bmatrix}a & a & b\end{bmatrix},\qquad\begin{bmatrix}a & b & a\end{bmatrix},\qquad\begin{bmatrix}b & a & a\end{bmatrix},
\end{equation}
其中
\begin{equation}
a=0.06308901449150222834033160287081916,\qquad1-2a,
\end{equation}
对应的权重为
\begin{equation}
w=\frac{0.05084490637020681692093680910686898}{2}.
\end{equation}
\item 第 4–6 号点的面积坐标为
\begin{equation}
\begin{bmatrix}a & a & b\end{bmatrix},\qquad\begin{bmatrix}a & b & a\end{bmatrix},\qquad\begin{bmatrix}b & a & a\end{bmatrix},
\end{equation}
其中
\begin{equation}
a=0.24928674517091042129163855310701908,\qquad b=1-2a,
\end{equation}
对应的权重为
\begin{equation}
w=\frac{0.11678627572637936602528961138557944}{2}.
\end{equation}
\item 第 7–12 号点的面积坐标为
\begin{equation}
\begin{gathered}\begin{bmatrix}a & b & c\end{bmatrix},\qquad\begin{bmatrix}a & c & b\end{bmatrix},\qquad\begin{bmatrix}b & a & c\end{bmatrix},\\
\begin{bmatrix}b & c & a\end{bmatrix},\qquad\begin{bmatrix}c & a & b\end{bmatrix},\qquad\begin{bmatrix}c & b & a\end{bmatrix},
\end{gathered}
\end{equation}
其中
\begin{equation}
\begin{aligned}a & =0.05314504984481694735324967163139815,\\
b & =0.31035245103378440541660773395655215,\\
c & =1-a-b,
\end{aligned}
\end{equation}
对应的权重为
\begin{equation}
w=\frac{0.08285107561837357519355345642044245}{2}.
\end{equation}
\end{itemize}
%

\subsection{十六点八阶积分器}

该积分器含十六个取值点:
\begin{itemize}[wide]
\item 第 1 号点的面积坐标为
\begin{equation}
\begin{bmatrix}a & a & a\end{bmatrix},\qquad a=\frac{1}{3},
\end{equation}
对应的权重为
\begin{equation}
w=\frac{0.14431560767778716825109111048906462}{2}.
\end{equation}
\item 第 2–4 号点的面积坐标为
\begin{equation}
\begin{bmatrix}a & a & b\end{bmatrix},\qquad\begin{bmatrix}a & b & a\end{bmatrix},\qquad\begin{bmatrix}b & a & a\end{bmatrix},
\end{equation}
其中
\begin{equation}
a=0.17056930775176020662229350149146450,\qquad b=1-2a,
\end{equation}
对应的权重为
\begin{equation}
w=\frac{0.10321737053471825028179155029212903}{2}.
\end{equation}
\item 第 5–7 号点的面积坐标为
\begin{equation}
\begin{bmatrix}a & a & b\end{bmatrix},\qquad\begin{bmatrix}a & b & a\end{bmatrix},\qquad\begin{bmatrix}b & a & a\end{bmatrix},
\end{equation}
其中
\begin{equation}
a=0.05054722831703097545842355059659895,\qquad b=1-2a,
\end{equation}
对应的权重为
\begin{equation}
w=\frac{0.03245849762319808031092592834178060}{2}.
\end{equation}
\item 第 8–10 号点的面积坐标为
\begin{equation}
\begin{bmatrix}a & a & b\end{bmatrix},\qquad\begin{bmatrix}a & b & a\end{bmatrix},\qquad\begin{bmatrix}b & a & a\end{bmatrix},
\end{equation}
其中
\begin{equation}
a=0.45929258829272315602881551449416932,\qquad b=1-2a,
\end{equation}
对应的权重为
\begin{equation}
w=\frac{0.09509163426728462479389610438858432}{2}.
\end{equation}
\item 第 11–16 号点的面积坐标为
\begin{equation}
\begin{gathered}\begin{bmatrix}a & b & c\end{bmatrix},\qquad\begin{bmatrix}a & c & b\end{bmatrix},\qquad\begin{bmatrix}b & a & c\end{bmatrix},\\
\begin{bmatrix}b & c & a\end{bmatrix},\qquad\begin{bmatrix}c & a & b\end{bmatrix},\qquad\begin{bmatrix}c & b & a\end{bmatrix},
\end{gathered}
\end{equation}
其中
\begin{equation}
\begin{aligned}a & =0.26311282963463811342178578628464359,\\
b & =0.00839477740995760533721383453929445,\\
c & =1-a-b,
\end{aligned}
\end{equation}
对应的权重为
\begin{equation}
w=\frac{0.02723031417443499426484469007390892}{2}.
\end{equation}
\end{itemize}
%

\section{四面体单元上的数值积分器}

为了使用 \ref{subsec:quadrature} 小节中介绍的通用方法,需要解析地算出标准四面体上的积分
\begin{equation}
\int_{0}^{1}\int_{0}^{1-\eta_{B}}\eta_{A}^{a}\,\eta_{B}^{b}\,\eta_{C}^{c}\dd{\eta_{A}}\dd{\eta_{B}}\dd{\eta_{C}}.
\end{equation}
此处引用一个更一般的结论:
\begin{equation}
\int_{0}^{1}\int_{0}^{1-\eta_{B}}\int_{0}^{1-\eta_{B}-\eta_{C}}\eta_{A}^{a}\,\eta_{B}^{b}\,\eta_{C}^{c}\,\eta_{D}^{d}\dd{\eta_{A}}\dd{\eta_{B}}\dd{\eta_{C}}=\frac{a!\,b!\,c!\,d!}{(a+b+c+d+3)!},
\end{equation}
其中 $\eta_{A},\eta_{B},\eta_{C},\eta_{D}$ 依次为 $A,B,C,D$ 四点对应的体积坐标。

证明过程与三角形上的结论类似,此处不再赘述。

\subsection{一点一阶积分器}

该积分器只含 1 个取值点,其体积坐标为
\begin{equation}
\begin{bmatrix}a & a & a & a\end{bmatrix},\qquad a=\frac{1}{4},
\end{equation}
对应的权重为
\begin{equation}
w=\frac{1}{6}.
\end{equation}


\subsection{四点二阶积分器}

该积分器含 4 个取值点,其体积坐标为
\begin{equation}
\begin{bmatrix}a & a & a & b\end{bmatrix},\qquad\begin{bmatrix}a & a & b & a\end{bmatrix},\qquad\begin{bmatrix}a & b & a & a\end{bmatrix},\qquad\begin{bmatrix}b & a & a & a\end{bmatrix},
\end{equation}
其中
\begin{equation}
a=0.13819660112501051517954131656343619,\qquad b-3a,
\end{equation}
对应的权重为
\begin{equation}
w=\frac{0.25}{6}.
\end{equation}


\subsection{十四点五阶积分器}

该积分器含 14 个取值点:
\begin{itemize}[wide]
\item 第 1–4 号点的体积坐标为
\begin{equation}
\begin{bmatrix}a & a & a & b\end{bmatrix},\qquad\begin{bmatrix}a & a & b & a\end{bmatrix},\qquad\begin{bmatrix}a & b & a & a\end{bmatrix},\qquad\begin{bmatrix}b & a & a & a\end{bmatrix},
\end{equation}
其中
\begin{equation}
a=0.31088591926330060979734573376345783,\qquad b=1-3a,
\end{equation}
对应的权重为
\begin{equation}
w=\frac{0.11268792571801585079918565233328633}{6}.
\end{equation}
\item 第 5–8 号点的体积坐标为
\begin{equation}
\begin{bmatrix}a & a & a & b\end{bmatrix},\qquad\begin{bmatrix}a & a & b & a\end{bmatrix},\qquad\begin{bmatrix}a & b & a & a\end{bmatrix},\qquad\begin{bmatrix}b & a & a & a\end{bmatrix},
\end{equation}
其中
\begin{equation}
a=0.09273525031089122640232391373703061,\qquad b=1-3a,
\end{equation}
对应的权重为
\begin{equation}
w=\frac{0.07349304311636194954371020548632750}{6}.
\end{equation}
\item 第 9–14 号点的体积坐标为
\begin{equation}
\begin{gathered}\begin{bmatrix}a & a & c & c\end{bmatrix},\qquad\begin{bmatrix}a & c & a & c\end{bmatrix},\qquad\begin{bmatrix}a & c & c & a\end{bmatrix},\\
\begin{bmatrix}c & a & a & c\end{bmatrix},\qquad\begin{bmatrix}c & a & c & a\end{bmatrix},\qquad\begin{bmatrix}c & c & a & a\end{bmatrix},
\end{gathered}
\end{equation}
其中
\begin{equation}
a=0.04550370412564964949188052627933943,\qquad c=0.5-a,
\end{equation}
对应的权重为
\begin{equation}
w=\frac{0.04254602077708146643806942812025744}{6}.
\end{equation}
\end{itemize}
%

\subsection{二十四点六阶积分器}

该积分器含 24 个取值点:
\begin{itemize}[wide]
\item 第 1–4 号点的体积坐标为
\begin{equation}
\begin{bmatrix}a & a & a & b\end{bmatrix},\qquad\begin{bmatrix}a & a & b & a\end{bmatrix},\qquad\begin{bmatrix}a & b & a & a\end{bmatrix},\qquad\begin{bmatrix}b & a & a & a\end{bmatrix},
\end{equation}
其中
\begin{equation}
a=0.21460287125915202928883921938628499,\qquad b=1-3a,
\end{equation}
对应的权重为
\begin{equation}
w=\frac{0.03992275025816749209969062755747998}{6}.
\end{equation}
\item 第 5–8 号点的体积坐标为
\begin{equation}
\begin{bmatrix}a & a & a & b\end{bmatrix},\qquad\begin{bmatrix}a & a & b & a\end{bmatrix},\qquad\begin{bmatrix}a & b & a & a\end{bmatrix},\qquad\begin{bmatrix}b & a & a & a\end{bmatrix},
\end{equation}
其中
\begin{equation}
a=0.04067395853461135311557944895641006,\qquad b=1-3a,
\end{equation}
对应的权重为
\begin{equation}
w=\frac{0.01007721105532064294801323744593686}{6}.
\end{equation}
\item 第 9–12 号点的体积坐标为
\begin{equation}
\begin{bmatrix}a & a & a & b\end{bmatrix},\qquad\begin{bmatrix}a & a & b & a\end{bmatrix},\qquad\begin{bmatrix}a & b & a & a\end{bmatrix},\qquad\begin{bmatrix}b & a & a & a\end{bmatrix},
\end{equation}
其中
\begin{equation}
a=0.32233789014227551034399447076249213,\qquad b=1-3a,
\end{equation}
对应的权重为
\begin{equation}
w=\frac{0.05535718154365472209515327785372602}{6}.
\end{equation}
\item 第 13–24 号点的体积坐标为
\begin{equation}
\begin{gathered}\begin{bmatrix}a & a & b & c\end{bmatrix},\qquad\begin{bmatrix}a & a & c & b\end{bmatrix},\qquad\begin{bmatrix}a & b & a & c\end{bmatrix},\qquad\begin{bmatrix}a & b & c & a\end{bmatrix},\\
\begin{bmatrix}a & c & a & b\end{bmatrix},\qquad\begin{bmatrix}a & c & b & a\end{bmatrix},\qquad\begin{bmatrix}b & a & a & c\end{bmatrix},\qquad\begin{bmatrix}b & a & c & a\end{bmatrix},\\
\begin{bmatrix}b & c & a & a\end{bmatrix},\qquad\begin{bmatrix}c & a & a & b\end{bmatrix},\qquad\begin{bmatrix}c & a & b & a\end{bmatrix},\qquad\begin{bmatrix}c & b & a & a\end{bmatrix},
\end{gathered}
\end{equation}
其中
\begin{equation}
\begin{aligned}a & =0.06366100187501752529923552760572698,\\
b & =0.60300566479164914136743113906093969,\\
c & =1-2a-b,
\end{aligned}
\end{equation}
对应的权重为
\begin{equation}
w=\frac{27/560}{6}.
\end{equation}
\end{itemize}
%

\subsection{四十六点八阶积分器}

该积分器含 46 个取值点:
\begin{itemize}[wide]
\item 第 1–4 号点的体积坐标为
\begin{equation}
\begin{bmatrix}a & a & a & b\end{bmatrix},\qquad\begin{bmatrix}a & a & b & a\end{bmatrix},\qquad\begin{bmatrix}a & b & a & a\end{bmatrix},\qquad\begin{bmatrix}b & a & a & a\end{bmatrix},
\end{equation}
其中
\begin{equation}
a=0.03967542307038990126507132953938949,\qquad b=1-3a,
\end{equation}
对应的权重为
\begin{equation}
w=\frac{0.00639714777990232132145142033517302}{6}.
\end{equation}
\item 第 5–8 号点的体积坐标为
\begin{equation}
\begin{bmatrix}a & a & a & b\end{bmatrix},\qquad\begin{bmatrix}a & a & b & a\end{bmatrix},\qquad\begin{bmatrix}a & b & a & a\end{bmatrix},\qquad\begin{bmatrix}b & a & a & a\end{bmatrix},
\end{equation}
其中
\begin{equation}
a=0.31448780069809631378416056269714830,\qquad b=1-3a,
\end{equation}
对应的权重为
\begin{equation}
w=\frac{0.04019044802096617248816115847981783}{6}.
\end{equation}
\item 第 9–12 号点的体积坐标为
\begin{equation}
\begin{bmatrix}a & a & a & b\end{bmatrix},\qquad\begin{bmatrix}a & a & b & a\end{bmatrix},\qquad\begin{bmatrix}a & b & a & a\end{bmatrix},\qquad\begin{bmatrix}b & a & a & a\end{bmatrix},
\end{equation}
其中
\begin{equation}
a=0.10198669306270330000000000000000000,\qquad b=1-3a,
\end{equation}
对应的权重为
\begin{equation}
w=\frac{0.02430797550477032117486910877192260}{6}.
\end{equation}
\item 第 12–16 号点的体积坐标为
\begin{equation}
\begin{bmatrix}a & a & a & b\end{bmatrix},\qquad\begin{bmatrix}a & a & b & a\end{bmatrix},\qquad\begin{bmatrix}a & b & a & a\end{bmatrix},\qquad\begin{bmatrix}b & a & a & a\end{bmatrix},
\end{equation}
其中
\begin{equation}
a=0.18420369694919151227594641734890918,\qquad b=1-3a,
\end{equation}
对应的权重为
\begin{equation}
w=\frac{0.05485889241369744046692412399039144}{6}.
\end{equation}
\item 第 17–22 号点的体积坐标为
\begin{equation}
\begin{gathered}\begin{bmatrix}a & a & c & c\end{bmatrix},\qquad\begin{bmatrix}a & c & a & c\end{bmatrix},\qquad\begin{bmatrix}a & c & c & a\end{bmatrix},\\
\begin{bmatrix}c & a & a & c\end{bmatrix},\qquad\begin{bmatrix}c & a & c & a\end{bmatrix},\qquad\begin{bmatrix}c & c & a & a\end{bmatrix},
\end{gathered}
\end{equation}
其中
\begin{equation}
a=0.06343628775453989240514123870189827,\qquad c=0.5-a,
\end{equation}
对应的权重为
\begin{equation}
w=\frac{0.03571961223409918246495096899661762}{6}.
\end{equation}
\item 第 23–34 号点的体积坐标为
\begin{equation}
\begin{gathered}\begin{bmatrix}a & a & b & c\end{bmatrix},\qquad\begin{bmatrix}a & a & c & b\end{bmatrix},\qquad\begin{bmatrix}a & b & a & c\end{bmatrix},\qquad\begin{bmatrix}a & b & c & a\end{bmatrix},\\
\begin{bmatrix}a & c & a & b\end{bmatrix},\qquad\begin{bmatrix}a & c & b & a\end{bmatrix},\qquad\begin{bmatrix}b & a & a & c\end{bmatrix},\qquad\begin{bmatrix}b & a & c & a\end{bmatrix},\\
\begin{bmatrix}b & c & a & a\end{bmatrix},\qquad\begin{bmatrix}c & a & a & b\end{bmatrix},\qquad\begin{bmatrix}c & a & b & a\end{bmatrix},\qquad\begin{bmatrix}c & b & a & a\end{bmatrix},
\end{gathered}
\end{equation}
其中
\begin{equation}
\begin{aligned}a & =0.02169016206772800480266248262493018,\\
b & =0.71993192203946593588943495335273478,\\
c & =1-2a-b,
\end{aligned}
\end{equation}
对应的权重为
\begin{equation}
w=\frac{0.00718319069785253940945110521980376}{6}.
\end{equation}
\item 第 35–46 号点的体积坐标为
\begin{equation}
\begin{gathered}\begin{bmatrix}a & a & b & c\end{bmatrix},\qquad\begin{bmatrix}a & a & c & b\end{bmatrix},\qquad\begin{bmatrix}a & b & a & c\end{bmatrix},\qquad\begin{bmatrix}a & b & c & a\end{bmatrix},\\
\begin{bmatrix}a & c & a & b\end{bmatrix},\qquad\begin{bmatrix}a & c & b & a\end{bmatrix},\qquad\begin{bmatrix}b & a & a & c\end{bmatrix},\qquad\begin{bmatrix}b & a & c & a\end{bmatrix},\\
\begin{bmatrix}b & c & a & a\end{bmatrix},\qquad\begin{bmatrix}c & a & a & b\end{bmatrix},\qquad\begin{bmatrix}c & a & b & a\end{bmatrix},\qquad\begin{bmatrix}c & b & a & a\end{bmatrix},
\end{gathered}
\end{equation}
其中
\begin{equation}
\begin{aligned}a & =0.20448008063679571424133557487274534,\\
b & =0.58057719012880922417539817139062041,\\
c & =1-2a-b,
\end{aligned}
\end{equation}
对应的权重为
\begin{equation}
w=\frac{0.01637218194531911754093813975611913}{6}.
\end{equation}
\end{itemize}
%

